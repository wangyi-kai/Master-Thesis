\documentclass{article}
\usepackage[UTF8]{ctex}
\usepackage{geometry}
\usepackage{graphicx}
\usepackage{subfigure}
\usepackage{diagbox}
\usepackage{amsthm}
\newtheorem{theorem}{Theorem}[section]
\newtheorem{lemma}[theorem]{Lemma} 
\geometry{a4paper,left=2cm,right=2cm,top=1cm,bottom=1cm}

\title{开题报告}
\author{王一开}
\date{\today}

\begin{document}
\maketitle

\section{选题背景和意义}
在数据驱动的机器学习取得进展之前,许多工程和物理领域采用的都是物理模型驱动,这些物理模型大多以偏微分方程的形式刻画或描述,例如流体力学中的Navier-Stokes方程组、电磁场理论中的Maxwell方程组、量子力学中的Schr$\ddot{o}$dinger方程组和相变问题中的Allen-Cahn方程等。通常情况下我们难以直接通过数学推导或者经验获得偏微分方程的精确描述,所以一般考虑求方程的近似解。数值方法是经典的偏微分方程解法,常用的数值方法主要有有限差分法、有限元法和有限体积方法,这些方法通过将给定的求解区域进行剖分,在剖分的网格上使用一系列基函数的组合进行逼近,最后得到方程的近似解。这些方法经过多年的发展,已经有了非常好的理论基础和实验结果,但是它们的求解严重依赖网格的剖分,网格划分的太粗求解精度会很低,而网格划分的太细又会带来巨大的计算代价和存储代价。除此之外,同一个数值方法求解不同的方程也有不同的过程,例如有限差分法求解不同的方程要考虑不同的网格划分格式,这个缺点也限制了数值方法的大规模应用。

随着数据和计算资源的爆炸式增长,机器学习与数据分析已经在多个领域例如计算机视觉,自然语言处理和生命科学产生了许多革命性的进展。然而很多时候在分析复杂的物理、生物或工程系统的过程中,数据采集的成本非常昂贵,我们不可避免地要面临在只有极其少量信息的情况下训练模型。在这种场景下,绝大多数先进的机器学习模型像深度神经网络、卷积神经网络等都会缺乏鲁棒性,而且模型的性能非常差。一部分研究人员考虑采用经典的机器学习算法去解决这些物理问题,经典的机器学习模型通常都是纯数据驱动的,通过建立一个从输入数据到输出数据的函数映射来训练一个有监督学习的机器学习模型,即依靠测量仪器收集到的历史数据学习一个具体的模型,模型性能的好坏与训练数据高度相关。然而这类纯数据驱动的算法却有很大的不足之处,第一纯数据驱动的算法缺乏解释性和适用性,针对不同的物理问题要考虑在模型中加入不同的物理限制,对于某个物理问题要设计针对该问题的特定模型去求解,而该模型无法应用于其他物理问题,这极大的限制了模型的适用范围;第二纯数据驱动的算法需要大量的训练数据使得模型收敛,而在许多物理和工程领域场景中,训练数据的获得代价非常昂贵,通常要面临在只有少量数据的情况下训练,此时训练出的模型往往泛化性能很差,无法准确预测物理参数;第三在物理和工程领域获得的数据常常隐含部分先验知识,如流体力学中的流场数据需要满足质量守恒和动量守恒定律,即满足Navier-Stokes方程组,而纯数据驱动的算法并为利用到这些知识,因此某种程度上并为充分利用训练数据。因此本文利用一种基于物理信息的神经网络\cite{raissi2019physics}(Physics-informed Neural Network,PINN),将数据驱动的机器学习和训练数据的先验知识结合在一起,能在少量训练数据的场景下,训练出自动满足物理约束的模型,在保证收敛的同时保持良好的泛化性能,能够准确预测需要的物理参数。
\section{研究内容}


\section{研究方法}

\section{研究计划}



\bibliographystyle{abbrv}
\bibliography{ref}
\end{document}